\begin{recipe}
[%
  portion = {\portion{6-8}}
]
{Jambalaya}
  \ingredients[19]{%
    1                             & onion (chopped) \\
    1                             & green bell pepper (chopped) \\
    \unit[2]{ribs}                & celery (chopped) \\
    \unit[1]{Tbsp}                & bacon fat \\
    \unit[1]{lb}                  & chicken (diced) \\
    \unit[12]{oz}                 & andoulle sausage (medallioned) \\
    \unit[\nicefrac{1}{2}]{cup}   & butter \\
    \unit[\nicefrac{1}{2}]{cup}   & flour \\
    \unit[5]{cans}                & chicken broth \\
    \unit[1]{can}                 & diced tomatoes \\
    \unit[\nicefrac{1}{2}]{Tbsp}  & chili powder \\
    \unit[1]{Tbsp}                & cumin \\
    \unit[1]{tsp}                 & poultry seasoning \\
    \unit[1]{Tbsp}                & cajun seasoning \\
    \unit[1]{cup}                 & Southern Comfort \\
    to taste                      & salt \\
    \unit[1\nicefrac{1}{2}]{cups} & rice \\
    \unit[1]{lb}                  & shrimp
  }
  \preparation{%
    \step Saut\'{e} onion, green pepper, and celery in the bacon fat with a pinch of salt
          over medium heat until tender.
    \step Remove vegetables from pot and set aside.
    \step Brown chicken and sausage, then remove from pot and set aside as well.
    \step Make a roux with the butter and flour. (Melt butter over medium heat, then
          add flour. Stir constantly until it is the color of peanut butter. \unit[~20]{mins})
    \step Add the chicken broth, tomatoes, chili powder, cumin, poultry seasoning,
          cajun seasoning, salt, Southern Comfort meat, and vegetables, then bring to a boil.
    \step Reduce heat, cover, and let simmer for 15-20 minutes.
    \step Add rice, cover, and continue slow simmer for 15 minutes.
    \step Add shrimp, cover, and simmer another 15 minutes.
  }
\end{recipe}
%%% Local Variables: 
%%% mode: latex
%%% TeX-master: "../master"
%%% End: 
