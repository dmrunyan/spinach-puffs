\documentclass[11pt, letterpaper]{article}

\usepackage{gensymb}
\usepackage{units}
\usepackage{xcookybooky}

\setRecipeLengths{
  preparationwidth = 0.5\textwidth,
  ingredientswidth = 0.45\textwidth
}

\begin{document}
  \begin{recipe}
  [%
    portion = {\portion{2-4}}
  ]
  {Guacamole}
    \ingredients[10]{%
      3                           & medium avocados \\
      \nicefrac{1}{2}             & medium onion \\
      2                           & beefsteak or roma tomatoes \\
      \unit[1]{Tbsp}              & cilantro \\
      1                           & jalape\~{n}o pepper \\
      \unit[2]{cloves}            & garlic \\
      \unit[\nicefrac{1}{2}]{tsp} & salt \\
      \unit[\nicefrac{1}{2}]{tsp} & cumin \\
      \unit[1]{tsp}               & cayenne pepper \\
      1                           & lime 
    }
    \preparation{%
      \step Seed and dice tomatoes into small pieces, then drain the excess juice
            using strainer or a paper towel. Mix with cilantro in a bowl and set aside.
      \step Mince onion. Seed and mince the jalape\~{n}o pepper. Mince and smash the
            garlic into a paste. Set aside.
      \step Seed and scoop the avocados into a mixing bowl, tossing them in a small amount
            of lime juice to avoid browning.
      \step Smash avocado with two wooden spoons, and fold in the rest of the ingredients,
            including about \unit[1]{Tbsp} of the lime juice.
      \step Push the guacamole into the bottom of the bowl to expose as little surface area
            to the air as possible. Using a paper towel, wipe any remaining bits of guacamole
            off the sides of the bowl that you couldn't scrape down into the bottom.
      \step Cover and seal with plastic wrap, and let sit for 1 hour before serving.
    }
    \hint{%
      Use as small of a mixing bowl for the final guacamole as you can get away with.
      This limits the amount of browning that begins to occur while the guacamole sits.
    }
  \end{recipe}
\end{document}